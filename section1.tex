\section{Notations préliminaires}

Cette section présente les notations et définitions de base utilisées dans le reste du document.

\subsection{Modélisation des systèmes}

\begin{definition}[Système de transition étiqueté (LTS)]
Pour représenter l'évolution des états d'un système, nous utilisons des systèmes de transition étiquetés (LTS - Labeled Transition Systems).
Soit $\prop$ l'ensemble des propositions sur un ensemble d'états $S$. Un LTS est un triplet $(S, \rightarrow, \lambda)$ où $\rightarrow \subseteq S \times A \times S$ est une relation de transition étiquetée par des actions issues d'un ensemble fini $A$, et $\lambda : S \to 2^\prop$ est une fonction de valuation qui associe à chaque état un ensemble de propriétés atomiques vraies dans cet état.
\end{definition}

\begin{definition}[Asset-Based System (ABS)]
\end{definition}

\subsection{Traces}

La littérature distingue deux sémantiques principales données aux traces utilisées pour modéliser les scénarios d'attaque dans les modèles de sécurité \cite{phillips1998graph, schneier1999attacktrees} : les traces basées sur les actions (ou action-based), comme définies dans \cite{vigo2014process} et les traces basées sur les états (ou state-based), comme présentées dans \cite{audinot2017correct, pinchinat2020library, audinot2018guided}.

\begin{definition}[Taces action-based)]
Une action-based sur les actions est une séquence finie d'actions $a_1 a_2 \ldots a_n$ où chaque action $a_i$ appartient à $\mathbb{B}$ l'ensemble des actions élémentaires de l'attaquant, issu de l'ensemble $\rightarrow$ des transitions du LTS.
\end{definition}

\begin{definition}[Traces state-based]
Une trace state-based est une séquence finie de vulations $\nu_1 \nu_2 \ldots \nu_n$ où chaque valuation $\nu_i$ est un ensemble de propositions atomiques vraies dans un état du LTS. 
\end{definition}

\begin{definition}[Concaténation synchrone]
\end{definition}

\begin{definition}[Shuffle de traces]
\end{definition}

\begin{definition}[Composition parallèle]
\end{definition}

\subsection{Attack trees}

\begin{definition}[Arbre d'attaque]
\end{definition}

\subsection{Libraries}

\begin{definition}[Bibliothèque d'arbres d'attaque]
\end{definition}
